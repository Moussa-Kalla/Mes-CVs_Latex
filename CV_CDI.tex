\documentclass[theme]{cv_einstein}
% Read cv_einstein.cls to look at all available options
\usepackage[utf8]{inputenc}
\usepackage[default]{raleway}
\usepackage{xcolor}
\usepackage{fontawesome}
\usepackage[french]{babel}
\usepackage{tikz}
% Caution: pargin=0cm means the CV won't print well.
% Using this template means that you accept it.
\usepackage[a4paper, portrait, margin=0cm]{geometry}
\usepackage{fontawesome}
\usepackage{array} % For better tabl formatting. See: https://tex.stackexchange.com/questions/12703/how-to-create-fixed-width-table-columns-with-text-raggedright-centered-raggedlef
\usepackage{enumitem} % See https://tex.stackexchange.com/a/199073/304372
\usepackage[pdftex, pdfauthor={Albert Einstein}, pdftitle={Albert Einstein, Public CV}, pdfsubject={CV of Albert Einstein},
pdfkeywords={Physics, Science, Applied Research, Fundamental Research}]
{hyperref}


\begin{document}
%------------------------------------------------------------------ Variables
% The left column contains the goals, summary, skills, etc.
% We define its width w.r.t. the width of the whole page
\newcommand{\lratio}{0.318}
\newlength{\leftcolwidth}
\setlength{\leftcolwidth}{\lratio\textwidth}
% The right column contains the main content, i.e. work experience, education, etc.
\newcommand{\rratio}{0.705}
\newlength{\rightcolwidth}
\setlength{\rightcolwidth}{\rratio\textwidth}
% Space to leave below a section, above the title of the following section
\newlength{\sectionspace}
\setlength{\sectionspace}{0.5cm}
% Space to leave below an item, above the following item
\newlength{\itemspace}
\setlength{\itemspace}{10pt}
% fbox stuff. You won't need to adjust these. You can safely ignore.
\setlength{\fboxrule}{0pt}
\setlength{\fboxsep}{4pt}
% Shortcuts to have table columns with fixed width AND positionning: [L]eft, [C]enter, [R]ight
\newcolumntype{L}[1]{>{\raggedright\let\newline\\\arraybackslash\hspace{0pt}}m{#1}}
\newcolumntype{C}[1]{>{\centering\let\newline\\\arraybackslash\hspace{0pt}}m{#1}}
\newcolumntype{R}[1]{>{\raggedleft\let\newline\\\arraybackslash\hspace{0pt}}m{#1}}
% Removes the (ugly) box around html links
\hypersetup{hidelinks}
%------------------------------------------------------------------
\title{Albert Einstein}
\author{\LaTeX{} Albert Einstein}
\date{1955}



    %-------------------------------------------------------------
    %-------------------------------------------------------------
    %-------------------------------------------------------------
    %                       UPPER PART
    %-------------------------------------------------------------
    %-------------------------------------------------------------
    %-------------------------------------------------------------

    %-------------------------------------------------------------
    %                       HEADER
    %-------------------------------------------------------------
    %Usage: \header{background-color}{name-color}{name}{title-color}{title}{summary-color}{summary}{portrait.jpg}{email@example.com}{phone}{country-flag.png}{city}{linkedin-id}
    \header
    {\textbf{Moussa Kalla}}
    {\textbf{Data Scientist · Quantitative Researcher}}
    { \normalsize
En tant que futur diplômé d'un Master spécialisé en Data Science et I.A,\\ \textbf{je suis à la recherche d'un stage de fin d'étude de six mois à partir de Mars 2024.}\\ Ma soif d'apprentissage demeure inébranlable et je suis prêt à relever de nouveaux défis tout en participant activement à des projets concrets. Ma créativité débordante et ma forte productivité font de moi un atout inestimable pour votre équipe.
    }
    {assets/portrait4.png}

    %-------------------------------------------------------------
    %                       CONTACT BAND
    %-------------------------------------------------------------
    % Usage: \contactband{background-color}{text-color}{email}{phone-number}{country-flag}{city}{linkedin-id}
    \contactband{moussa.kalla.am@gmail.com }{+33 744885914}{assets/france2.jpg}{Poitiers, France}{Mon LinkedIn}{Mon Github}

    \vspace{\headerheight} % The header is only a TIKZ image. We must give it space to appear and not be hidden by what comes next.

    \setlength{\columnsep}{0px}
    \columnratio{\lratio}
    \begin{paracol}{2}
        \paracolbackgroundoptions

        
        \begin{leftcolumn}
        
        {\color{white}    
        \noindent \footnotesize
           \footnotesize\color{white}
            \heading{\faCogs}{Mes Compétences}
            \begin{minipage}[c]{\leftcolwidth}
                \begin{tabular}{c}
                    \bubblediagram{
                    % Usage: \bubblediagram{list of comma-separated text items}
                    % The first item will be written in the main bubble, at the center of the diagram
                    % All other items will be written in their own satellite bubble
                        % Main bubble
                        {\textbf{\;\;Programmation} \\ \textbf{\&}\\
                        \textbf{Logiciels}},
                        \textbf{Python \& C++},
                        \textbf{SQL \& R},
                        \textbf{Git \& GitLab},
                        \textbf{TensorFlow}\\ \textbf{\& PyTorch},
                        \textbf{POWER BI},
                        \textbf{\;Matlab\;}
                       } 
                \end{tabular}
            \end{minipage}
\begin{minipage}[c]{\leftcolwidth}
                \begin{tabular}{c}
                    \bubblediagram{
                    % Usage: \bubblediagram{list of comma-separated text items}
                    % The first item will be written in the main bubble, at the center of the diagram
                    % All other items will be written in their own satellite bubble
                        % Main bubble
                        {\textbf{Data Science} \\ \textbf{\&}\\
                        \textbf{Machine Learning}},
                        % Satellites
                        \textbf{Data Analysis},
                        \textbf{Database} \\  \textbf{Management} \\  \textbf{System},
                        \textbf{Deep Learning},
                        \textbf{Data} \\ \textbf{Visualization},
                        \textbf{Data Mining}
                       } 
                \end{tabular}
            \end{minipage}

\heading{\faStar}{SOFTSKILLS }
            \begin{minipage}[c]{\leftcolwidth}
                \begin{tabular}{r|l}
                     Capacité d'adaptation & \pictofraction{4}\\ [0.04em]
                    Esprit d'équipe & \pictofraction{4}\\ [0.04em]
                    Dynamisme & \pictofraction{4}
                \end{tabular}
            \end{minipage}
        }
        \end{leftcolumn}
         \begin{leftcolumn} \noindent \footnotesize
        {\color{white}
            %-------------------------------------------------------------
            %                       LANGUAGES
            %-------------------------------------------------------------
              % To leave a margin with the top of the page
            \heading{\faLanguage}{Langues}
            \begin{minipage}[r]{\leftcolwidth}
                \begin{tabular}{r|l}
                    \textbf{Haoussa} &    \textbf{Langue maternelle} \\[0.03em]
                    \textbf{Français} & \textbf{Bilingue}\\[0.03em]
                    \textbf{Anglais}  & \textbf{Intermédiaire}
                \end{tabular}
            \end{minipage}
        }
        \end{leftcolumn}

        \begin{rightcolumn}\noindent \small
            \hspace{-2.4pt}\heading{\faSuitcase}{Expériences professionnels }
            \cvevent{2024}{Mars - Sept}{CDD : Data Science - Transformation Numerique}{SAFT (TotalEnergies)}{Poitiers, France}{assets/Saft.png}
            {$\bullet$ Mise en place d’une ressource Azure Data Factory pour l'extraction, la transformation et  l’ingestion des données de la division Aerospace Défense et
            Performance.\: \:\:\: \:\:\: \:\:\: \:\:\: \:\:\: \:\:\: \:\:\: \:\:\: \:\:\: \:\:\: \:\:\: \:\:\: \:\:\: \:\:\: \:\:\: \:\:\: \:\:\: \:\:\: \:\:\: \:\:\: \:\:\: \:\:\: \:\:\: \:\:\: \:\:\: \:\:\: \:\:\: \:\:\: \:\:\: \:\:\: \:\:\: \:\:\: \:\:\: \:\:\: \:\:\: 
            $\bullet$ Création et maintenance de tableaux de bord Power BI standards pour toute la division.
            \: \:\:\: \:\:\: \:\:\: \:\:\: \:\:\: \:\:\: \:\:\: \:\:\: \:\:\: \:\:\: \:\:\: \:\:\: \:\:\: \:\:\: \:\:\: \:\:\: \:\:\: \:\:\: \:\:\: \:\:\: \:\:\: \:\:\: \:\:\: \:\:\: \:\:\: \:\:\: \:\:\: \:\:\: \:\:\: \:\:\: \:\:\: \:\:\: \:\:\: \:\:\: \:\:\:
            $\bullet$ Développement et mise en production de modèles de Machine Learning et LLM sur Azure Machine Learning Studio}
            \vspace{0.01cm}\\
            \cvevent{2024}{Mars - Sept}{CDD : Data Science - Digitalisation}{McCormick \& Company}{Avignon France}{assets/McCormick.png}
            {$\bullet$ Développement rapports de Power BI pour une meilleure visualisation des données commerciales et logistiques.
            \: \:\:\: \:\:\: \:\:\: \:\:\: \:\:\: \:\:\: \:\:\: \:\:\: \:\:\: \:\:\: \:\:\: \:\:\: \:\:\: \:\:\: \:\:\: \:\:\: \:\:\: \:\:\: \:\:\: \:\:\: \:\:\: \:\:\: \:\:\: \:\:\: \:\:\: \:\:\: \:\:\: \:\:\: \:\:\: \:\:\: \:\:\: \:\:\: \:\:\: \:\:\: \:\:\:
            $\bullet$ Automatisation d’une pipeline d’extraction de données depuis SAP.}
            \vspace{0.01cm}\\
            \cvevent{2024}{Mars - Sept}{Stage Ingénieur - Data Science \& Modelisation }{ArcelorMittal}{Dunkerque, France}{assets/arcelormittal_.jpg}
            {$\bullet$ Conception de modèles de prédiction pour l’optimisation de la détection des incidents dans la coulée continue, en
            analysant et en modélisant des données de température en temps réel. Réduction de 60\% du nombre d’incidents.
            
            $\bullet$ Développement d’outils automatisés pour la visualisation des données, facilitant ainsi l’analyse des incidents.}

        \end{rightcolumn}

        
          \begin{rightcolumn}\noindent \small
            \hspace{-2.4pt}\heading{\faGraduationCap}{Diplômes et Formations}
            \cvevent{2024 - 2025}{}{Master 2 - Data Science \& IA }{IA School\: \:\:\: \:\:\: \:\:\: \:\:\: \:\:\: \:\:\: \:\:\: \:\:\: \:\:\: \:\:\: \:\:\: \:\:\: \:\:\: \:\:\: \:\:\: \:\:\: \:\:\: \:\:\: \:\:\: \:\:\: \:\:\: \:\:\: \:\:\: \:\:\: \:\:\: \:\:\: \:\:\: \:\:\: \:\:\: \:\:\: \:\:\: \:\:\: \:\:\: \:\:\: \:\:\: \:\:\: \:\:Groupe GEMA}{Paris, France}{assets/IASchool.jpg}
            {$\bullet$  Data Management,  Distributed Machine Learning,  DevOps-MLOps,  Cloud Computing.\: \:\:\: \:\:\: \:\:\: \:\:\: \:\:\: \:\:\: \:\:\: \:\:\: \:\:\: \:\:\: \:\:\: \:\:\: \:\:\: \:\:\: \:\:\: \:\:\: \:\:\: \:\:\: \:\:\: \:\:\: \:\:\: \:\:\: \:\:\: \:\:\: \:\:\: \:\:\: \:\:\: \:\:\: \:\:\: \:\:\: \:\:\: \:\:\: \:\:\: \:\:\: \:\:\: \:\:\: \:\: $\bullet$ Business Intelligence, Analyse des besoins \& Stratégie Big Data, Management de projet.}
              \vspace{0.005cm}\\
            \cvevent{2023 - 2024}{}{Master 2 - Ingénierie Mathématique \& Informatique}{École d’ingénieur du Littoral cote d’Opale\: \:\:\: \:\:\: \:\:\: \:\:\: \:\:\: \:\:\: \:\:\: \:\:\: \:\:\: \:\:\: \:\:\: \:\:\: \:\:\: \:\:\: \:\:\: \:\:\: \:\:\: \:\:\: \:\:\: \:\:\: \:\:\: \:\:\: \:\:\: \:\:\: \:\:\: \:\:\: \:\:\: \:\:\: \:\:\: \:\:\: \:\:\: \:\:\: \:\:\: \:\:\: \:\:\: \:\:\: \:\:Université du Littoral Côte d'Opale}{Calais, France}{assets/Eilco5.jpg}
            {$\bullet$ Big data, Analyse des données, Machine Learning,
             Deep Learning.\: \:\:\: \:\:\: \:\:\: \:\:\: \:\:\: \:\:\: \:\:\: \:\:\: \:\:\: \:\:\: \:\:\: \:\:\: \:\:\: \:\:\: \:\:\: \:\:\: \:\:\: \:\:\: \:\:\: \:\:\: \:\:\: \:\:\: \:\:\: \:\:\: \:\:\: \:\:\: \:\:\: \:\:\: \:\:\: \:\:\: \:\:\: \:\:\: \:\:\: \:\:\: \:\:\: \:\:\: \:\: $\bullet$ Admis dans cette école via un programme d'échange.}
              \vspace{0.005cm}\\
            \cvevent{2022 - 2023}{}{Master 1 - Ingénierie Mathématique \& Statistique}{Faculté des Sciences \: \:\:\: \:\:\: \:\:\: \:\:\: \:\:\: \:\:\: \:\:\: \:\:\: \:\:\: \:\:\: \:\:\: \:\:\: \:\:\: \:\:\: \:\:\: \:\:\: \:\:\: \:\:\: \:\:\: \:\:\: \:\:\: \:\:\: \:\:\: \:\:\: \:\:\: \:\:\: \:\:\: \:\:\: \:\:\: \:\:\: \:\:\: \:\:\: \:\:\: \:\:\: \:\:\: \:\:\: \:\: Université Mohammed V}{Rabat, Maroc}{assets/FSR3.jpg}
            {$\bullet$ Modélisation stochastique, Optimisation avancée, Programmation mathématique. .\: \:\:\: \:\:\: \:\:\: \:\:\: \:\:\: \:\:\: \:\:\: \:\:\: \:\:\: \:\:\: \:\:\: \:\:\: \:\:\: \:\:\: \:\:\: \:\:\: \:\:\: \:\:\: \:\:\: \:\:\: \:\:\: \:\:\: \:\:\: \:\:\: \:\:\: \:\:\: \:\:\: \:\:\: \:\:\: \:\:\: \:\:\: \:\:\: \:\:\: \:\:\: \:\:\: \:\:\: \:\: $\bullet$ Ce Master est hautement sélectif au sein du royaume et offre une formation de\:\:\: \:\:\:pointe en mathématiques fondamentales et appliquées.}
               \vspace{0.005cm}\\
            \cvevent{2018 - 2022}{}{Diplôme de licence - Génie Mathématique}{Faculté des Sciences et Techniques \: \:\:\: \:\:\: \:\:\: \:\:\: \:\:\: \:\:\: \:\:\: \:\:\: \:\:\: \:\:\: \:\:\: \:\:\: \:\:\: \:\:\: \:\:\: \:\:\: \:\:\: \:\:\: \:\:\: \:\:\: \:\:\: \:\:\: \:\:\: \:\:\: \:\:\: \:\:\: \:\:\: \:\:\: \:\:\: \:\:\: \:\:\: \:\:\: \:\:\: \:\:\: \:\:\: \:\:\: \:\: Université Sultan Moulay Slimane}{Béni Mellal, Maroc}{assets/fstbm1.jpg}
            {$\bullet$ Statistique inférentielle et multivariée,  Probabilités avancées,  Théorie de la mesure. \:\:\:\:\:\:\:\:\:\:\:\:\:\:\:\:\:\:\:\:\:\: \:\:\:\:\:\:\:\:\:\:\:\:\:\:\:\:\:\:\:\:\:$\bullet$ Projet de fin d'étude : Approximation des équations stochastiques }
              \vspace{0.005cm}\\
            \cvdevent{2015 - 2018}{}{Baccalauréat - Série scientifique D}{Lycée Issa Korombé}{Niamey, Niger}{}
            {$\bullet$ Premier de mon lycée.}}
            \end{rightcolumn}

          
            
        
        %-------------------------------------------------------------
        %-------------------------------------------------------------
        %                       LEFT COLUMN
        %-------------------------------------------------------------
        %-------------------------------------------------------------
       
        %-------------------------------------------------------------
        %-------------------------------------------------------------
        %                       RIGHT COLUMN
        %-------------------------------------------------------------
        %-------------------------------------------------------------
        
\begin{leftcolumn} \noindent \small
    \heading{\faHeartbeat}{Mes Hobbies}
    \textcolor{white}{\: \faLineChart\ Bourse }
    \textcolor{white}{\:\faSoccerBallO \ Football} \textcolor{white}{\: \faGamepad\ Jeux vidéo}  \\
    \textcolor{white}{\: \faBook\ Lecture}
    \textcolor{white}{\: \faLeaf\ Écologie}
    \textcolor{white}{\: \faPlane\ Voyages }
\end{leftcolumn}
        \begin{leftcolumn}\noindent \footnotesize
        {\color{white}
           \heading{\faTrophy}{mes distinctions}
\normalsize 
 $\Longrightarrow$ Boursier d'excellence de l'ICESCO \\  (2023 - 2024) \\ [0.5em]
 $\Longrightarrow$  Boursier d'excellence de coopération \\ \;\;\;\; Niger - Maroc  (2018 - 2022) \\ [0.5em]
 $\Longrightarrow$ Boursier de l'AMCI (2018 - 2023)

}
        \end{leftcolumn}
        %-------------------------------------------------------------
        %-------------------------------------------------------------
        %                       RIGHT COLUMN
        %-------------------------------------------------------------
        %-------------------------------------------------------------
        \begin{rightcolumn}\noindent \small
            \hspace{-2.4pt}\heading{\faTv}{CERTIFICATIONS}
            \cvcevent{2023}{Sept - oct}{Azure Data Fundamentals}{E-Learning}{}{assets/azure-data.png}
            {$\bullet$ Cette formation aborde le modèle de données dimensionnelles distribuées dynamiques (D4M), qui fusionne la théorie des graphes, l'algèbre linéaire et les\:\:\:\:\:\:\:\:\:\:\:\:\:\:\:\:\:\:\:\:\: bases de données pour résoudre les défis liés aux Big Data.}
             \vspace{0.005cm}\\
            \cvcevent{2023}{Sept - oct}{Azure AI Fundamentals}{E-Learning}{}{assets/azure-ai.png}
            {$\bullet$ Cette formation aborde le modèle de données dimensionnelles distribuées dynamiques (D4M), qui fusionne la théorie des graphes, l'algèbre linéaire et les\:\:\:\:\:\:\:\:\:\:\:\:\:\:\:\:\:\:\:\:\: bases de données pour résoudre les défis liés aux Big Data.}
            \end{rightcolumn}
            \begin{rightcolumn}\noindent \small
        \end{rightcolumn}
        \vspace{0em}
    \end{paracol}
\end{document}
